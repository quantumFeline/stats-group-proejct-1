\documentclass{article}
\usepackage[utf8]{inputenc}
\usepackage[margin=1in]{geometry}
%\usepackage{polski}
\usepackage{amsmath}
\usepackage{amssymb}
\usepackage{amsthm}
\usepackage{gensymb}
\usepackage{wasysym}
\usepackage{graphicx}
\usepackage{subcaption}
\usepackage[makeroom]{cancel}
\usepackage{xcolor}
\usepackage{hyperref}
\usepackage{enumitem}
\usepackage{listings}

\title{Statistical Data Analysis 2 - Final Project - Report}
\author{Czapiewska Magdalena, Khodina Anastasiya, Nachaieva Veronika, Znamierowski Mikołaj}

\begin{document}

\maketitle

\tableofcontents

% You should prepare and submit a comprehensive report describing your experimental setup, the datasets generated, and the results obtained. The report should include the specifications of all Boolean networks considered in your study, along with the corresponding input files containing the simulated datasets provided to BNFinder2. It should justify all methodological choices made and present the results both in written form and through appropriate graphical representations. Finally, the conclusions drawn from your experiments should be clearly articulated.

\section{Boolean Network representation and construction}
TODO: I think that here we should have a description of BooleanNetworks.py

\section{Attractor detection implementation}
\subsection{Attractors in synchronous mode}
\subsection{Attractors in asynchronous mode}

\section{Datasets generation}

\subsection{The way of generating a dataset with given characteristics}

\subsection{Description of the datasets generated}

\section{BNFinder2 usage}

\section{Evaluation of the accuracy of the reconstructed network}

\subsection{Evaluation metrics chosen}

\subsection{Results with respect to the characteristics of the datasets and the scoring functions used}

\section{Here something about part II - I'm not sure what titles are suitable}
TODO: Please change the title of this section and maybe create several sections if needed

\section{Conclusions drawn from experiments}

\section{Contributions}

Authors (given in alphabetical order) and their contributions to the project:
\begin{itemize}
    \item Czapiewska Magdalena:
    \begin{itemize}
        \item mmm
        \item mmm
    \end{itemize}
    \item Khodina Anastasiya:
    \begin{itemize}
        \item mmm
        \item mmm
    \end{itemize}
    \item Nechaieva Veronika:
    \begin{itemize}
        \item Organisational activity (Github etc.)
        \item Reworking the code for Boolean networks; readability
        \item Saving and loading the network ground truth structure
        \item Reworking the saving dataset function into a proper format for BNFinder2
        \item Converting the Python scripts to work with command-line arguments
        \item Testing BNF base functionality
    \end{itemize}
    \item Znamierowski Mikołaj:
    \begin{itemize}
        \item Finding and sharing crucial resources regarding BNFinder2.
        \item Initiating the work.
        \item Implementing the code for Boolean network: setting proper structure, functionalities, transitions, etc.
        \item Implementing functions that create the datasets and save them in a proper format for BNFinder2.
        \item Explaining to the rest of the group how to use the code.
    \end{itemize}
\end{itemize}

\end{document}
